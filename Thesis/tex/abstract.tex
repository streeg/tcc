% O \emph{abstract} é o resumo feito na língua Inglesa. Embora o conteúdo apresentado
% deva ser o mesmo, este texto não deve ser a tradução literal de cada palavra ou
% frase do resumo, muito menos feito em um tradutor automático. É uma língua
% diferente e o texto deveria ser escrito de acordo com suas nuances (aproveite para ler
% \url{http://dx.doi.org/10.6061%2Fclinics%2F2014(03)01}). Por exemplo: \emph{This work presents useful information on how to create a scientific text to describe
% and provide examples of how to use the Computer Science Department's \LaTeX\ class. The \unbcic\
% class defines a standard format for texts, simplifying the process of generating
% CIC documents and enabling authors to focus only on content. The standard was approved
% by the Department's professors and used to create this document. Future work includes
% continued support for the class and improvements on the explanatory text.}

This study introduces a innovative approach to enhance the detection of vulnerabilities in Java cryptographic APIs,
aiming to strengthen the security of applications built on this technology. By integrating the tools CogniCrypt 
and CryptoGuard with LibScout, we enable the precise identification of the source of warnings related to external libraries.
This qualitative approach represents a significant advancement in promoting security in Java applications, 
contributing to a more resilient digital ecosystem protected against potential cyber threats. 
The incorporation of warning source identification also allows for direct suggestions to library developers, 
streamlining the vulnerability correction process. However, we encountered challenges when analyzing obfuscated 
code and utilizing clusters and datasets in LibScout, highlighting the need for improvements in this tool. 
The integration proposed in this work represents a significant step towards comprehensive security for sensitive 
data and critical systems in Java applications.