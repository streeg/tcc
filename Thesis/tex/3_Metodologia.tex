\section{Hipótese de Trabalho}

A hipótese deste estudo é: Ao integrar os resultados do LibScout aos contextos das ferramentas CryptoGuard e CogniCrypt, será possível não apenas detectar potenciais vulnerabilidades em APIs criptográficas, mas também identificar as correspondências associadas a bibliotecas externas. Esta integração proporciona uma abordagem mais abrangente e eficaz para a segurança de aplicações Java que utilizam operações criptográficas.

\section{Metodologia}

\subsection{Estrutura da Metodologia (GQM)}

Conforme discutido no Capítulo 1, os objetivos desta pesquisa incluem a identificação e análise de vulnerabilidades em APIs criptográficas Java, com foco na integração dos resultados obtidos pelo LibScout aos contextos fornecidos pelas ferramentas CryptoGuard e CogniCrypt. Para alcançar esses objetivos, adotamos a abordagem GQM (Goal, Question, Metric), conforme descrito a seguir:

\subsubsection{Objetivo}
Adaptar ferramentas de análise estática para identificar se os \textit{warnings} são provenientes do código de origem da aplicação ou de bibliotecas externas.

\subsubsection{Questões de Pesquisa}
Para responder às questões propostas, definimos as seguintes métricas baseadas nos dados coletados:

\begin{itemize}
\item \textbf{RQ1:} Qual é a quantidade de \textit{warnings} no \textit{dataset} de aplicativos e qual a prevalência de \textit{crypto API misuses} em aplicativos Android?
\textbf{Resposta:} \\
A análise incluiu 307 aplicativos Android de seis categorias diferentes do repositório F-Droid. As ferramentas CogniCrypt e CryptoGuard identificaram um total de 11.036 \textit{warnings} de uso indevido de criptografia (\textit{misuses}) no CogniCrypt e 4.964 \textit{warnings} no CryptoGuard. A maior parte dos \textit{warnings} foi encontrada nas categorias de Internet e Finanças. A categoria Internet concentrou 30,87\% dos \textit{warnings} no CogniCrypt, enquanto as categorias Finanças e Conectividade concentraram 56,6\% dos \textit{warnings} no CryptoGuard. Esses resultados indicam que a prevalência de \textit{crypto API misuses} é significativa, especialmente em determinadas categorias de aplicativos.

\item \textbf{RQ2:} Qual a prevalência de \textit{crypto API misuses} específicas de bibliotecas externas?

\textbf{Resposta:} \\
Após a integração dos resultados do LibScout com os \textit{warnings} identificados pelo CogniCrypt e CryptoGuard, foi possível diferenciar os \textit{warnings} associados a bibliotecas externas daqueles associados a bibliotecas nativas. O estudo revelou que, no CogniCrypt, de um total de 6.798 bibliotecas analisadas, 1.710 eram externas e 726 eram potencialmente externas. No CryptoGuard, de um total de 2.710 bibliotecas, 1.149 eram externas e 245 eram potencialmente externas. Isso demonstra que uma parte considerável dos \textit{warnings} detectados pode ser atribuída a bibliotecas externas, o que sugere a eficácia da integração dessas ferramentas na identificação da origem dos \textit{warnings} e na detecção de vulnerabilidades específicas de bibliotecas externas.

\item \textbf{RQ3:} Como a ofuscação de código afeta a detecção de vulnerabilidades em APIs criptográficas?

\textbf{Resposta:} \\
Durante o experimento, enfrentamos dificuldades significativas ao tentar identificar bibliotecas em códigos ofuscados. A ofuscação impede que as ferramentas como LibScout reconheçam as bibliotecas externas, resultando em uma falha na identificação correta dessas bibliotecas. Consequentemente, o script que realiza o casamento entre as vulnerabilidades detectadas pelas ferramentas de análise estática (CogniCrypt e CryptoGuard) e os resultados do LibScout frequentemente resultava em resultado negativo. Isso indica que a ofuscação de código afeta diretamente a capacidade de detecção de vulnerabilidades em APIs criptográficas, particularmente na identificação de bibliotecas externas. Esse impacto negativo reforça a necessidade de desenvolver técnicas mais robustas para lidar com código ofuscado, a fim de garantir uma análise de segurança mais precisa e abrangente.

\item \textbf{RQ4:} Qual a diferença na detecção de vulnerabilidades entre bibliotecas nativas e externas?

\textbf{Resposta:} \\
A análise revelou diferenças notáveis na detecção de vulnerabilidades entre bibliotecas nativas e externas, dependendo da ferramenta utilizada. No CogniCrypt, a maioria das vulnerabilidades foi detectada em bibliotecas nativas: de um total de 6.798 bibliotecas analisadas, 4.542 eram nativas, enquanto 1.710 eram externas e 726 eram potencialmente externas. Isso indica que, para o CogniCrypt, as vulnerabilidades são mais prevalentes em bibliotecas nativas.

Por outro lado, no CryptoGuard, a distribuição entre vulnerabilidades em bibliotecas nativas e externas foi mais equilibrada. De um total de 2.710 bibliotecas analisadas, 1.316 eram nativas, 1.149 eram externas e 245 eram potencialmente externas. Este resultado mostra que, para o CryptoGuard, as vulnerabilidades estão quase igualmente distribuídas entre bibliotecas nativas e externas.

Esses resultados sugerem que, enquanto o CogniCrypt tende a identificar mais vulnerabilidades em bibliotecas nativas, o CryptoGuard detecta uma quantidade quase igual de vulnerabilidades tanto em bibliotecas nativas quanto externas. Isso pode indicar diferenças na abordagem das ferramentas em relação à análise de bibliotecas externas e reforça a necessidade de considerar ambas as fontes de vulnerabilidades ao avaliar a segurança de uma aplicação.
\end{itemize}

\subsubsection{Métricas}
Para responder as questões RQ1 e RQ2, as seguintes métricas foram estabelecidas:

\begin{itemize}
\item \textbf{M1: Número total de warnings detectados.} \
Esta métrica quantifica o total de \textit{warnings} (alertas de vulnerabilidade) identificados pelas ferramentas de análise estática, CogniCrypt e CryptoGuard, nos aplicativos analisados. O estudo identificou 11.036 \textit{warnings} com o CogniCrypt e 4.964 com o CryptoGuard. Essa métrica é essencial para quantificar a presença geral de vulnerabilidades no conjunto de dados de aplicativos Android.
\subsubsection{Métricas}
Para responder as questões RQ1 e RQ2, as seguintes métricas foram estabelecidas:

\begin{itemize}
\item \textbf{M1: Número total de warnings detectados.} \
Esta métrica quantifica o total de \textit{warnings} (alertas de vulnerabilidade) identificados pelas ferramentas de análise estática, CogniCrypt e CryptoGuard, nos aplicativos analisados. O estudo identificou 11.036 \textit{warnings} com o CogniCrypt e 4.964 com o CryptoGuard. Essa métrica é essencial para quantificar a presença geral de vulnerabilidades no conjunto de dados de aplicativos Android.

\end{itemize}

\subsection{Fases da Pesquisa}

Essa pesquisa foi organizada em várias fases, conforme a Figura Y, para estruturar e direcionar o estudo:

\begin{itemize}
\item \textbf{Identificação e Seleção das Ferramentas:} Inicialmente, realizamos uma análise da literatura para identificar ferramentas adequadas para a detecção de vulnerabilidades em APIs criptográficas. Consideramos o uso de LibRadar e LibScout, mas optamos por utilizar o LibScout devido à sua maior atualização e capacidade de identificar bibliotecas externas de forma mais precisa.

\item \textbf{Integração das Ferramentas:} Após a seleção, integramos os resultados do LibScout aos contextos das ferramentas CryptoGuard e CogniCrypt. Essa integração permitiu uma análise mais detalhada e contextualizada das vulnerabilidades encontradas, especialmente em bibliotecas externas.

\item \textbf{Execução do Experimento:} O experimento foi realizado em várias etapas, incluindo a caracterização do dataset, a coleta de dados, e a análise dos resultados. Durante a análise, geramos gráficos e tabelas para ilustrar a distribuição de \textit{warnings} e a prevalência de vulnerabilidades, conforme descrito no próximo capítulo.

\end{itemize}

\subsection{Diagrama da Integração}

Para uma melhor visualização do processo, o diagrama da Figura Z mostra a integração entre o LibScout e as ferramentas CogniCrypt e CryptoGuard, detalhando como os resultados foram processados e analisados.