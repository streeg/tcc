\section{Conclusão}

Este estudo se concentrou na avaliação comparativa entre as ferramentas CogniCrypt, CryptoGuard e LibScout para melhorar a detecção de vulnerabilidades em APIs criptográficas Java. As descobertas evidenciam que a integração destas ferramentas aprimora significativamente a precisão e a eficácia na identificação de falhas de segurança, oferecendo uma abordagem holística e mais robusta para a segurança de aplicações Java.

A ferramenta CryptoGuard identificou menos bibliotecas externas em comparação com o CogniCrypt, porém, ambas as ferramentas trouxeram resultados favoráveis para a detecção da origem do código malicioso. 

As implicações destas descobertas podem ser úteis para a comunidade de desenvolvedores Java. A utilização integrada destas ferramentas pode contribuir para as práticas atuais de desenvolvimento seguro, permitindo uma identificação mais rápida e precisa de vulnerabilidades. Isso não apenas melhora a segurança das aplicações, mas também otimiza o processo de desenvolvimento, economizando tempo e recursos. \cite{perception_developers}

O estudo enfrentou limitações, como a complexidade na análise de código obfuscado \cite{api_misuses_zhang}, que impactam a eficácia das ferramentas. Estas limitações destacam a necessidade contínua de aprimoramento na tecnologia de detecção de vulnerabilidades, reforçando a importância de abordagens adaptativas e inovadoras na segurança cibernética.

Para pesquisas futuras, sugere-se o desenvolvimento de metodologias mais avançadas e aprimoramento das ferramentas existentes para abordar novos desafios de segurança. A expansão do escopo para outras linguagens de programação e plataformas pode oferecer uma contribuição mais abrangente para a segurança de aplicações. Também é recomendado um set de dados mais amplo e diversificado para avaliar a eficácia das ferramentas.

A segurança em APIs criptográficas Java é de suma importância no cenário digital atual. Este estudo contribui para este campo, oferecendo insights valiosos e abrindo caminho para futuras inovações. A necessidade de pesquisa contínua e desenvolvimento de novas soluções de segurança é clara, dada a evolução constante das ameaças cibernéticas.
