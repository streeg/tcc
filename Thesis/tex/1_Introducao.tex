\section{Introdução}%

A criptografia, uma disciplina essencial da segurança da informação, é fundamental para proteger sistemas digitais e dados sensíveis de ameaças cibernéticas. \cite{what_is_cryptography} Com a complexidade das aplicações aumentando e a variedade de bibliotecas e frameworks disponíveis, o desenvolvimento de ferramentas automatizadas capazes de detectar possíveis falhas e vulnerabilidades nas interfaces de programação de aplicações (APIs) criptográficas torna-se crucial. \cite{api_misuses_zhang}

O surgimento da linguagem CrySL permitiu a definição precisa de regras para o uso seguro de APIs criptográficas em código Java. \cite{CogniCrypt} A linguagem permitiu a criação de padrões mais rigorosos para a implementação de métodos de criptografia seguros. No entanto, há um grande número de investigações sobre as dúvidas sobre a eficácia das ferramentas atuais e a precisão de seus alertas.

Neste contexto, o presente estudo empreende uma análise qualitativa abrangente da detecção de vulnerabilidades em APIs criptográficas, valendo-se das ferramentas CogniCrypt \cite{CogniCrypt} e CryptoGuard \cite{CryptoGuard}.

Deparamo-nos com a complexidade inerente à análise de código obfuscado durante a realização deste estudo, o que reforçou o valor de considerar uma variedade de contextos de implementação ao avaliar a eficácia das ferramentas de detecção de vulnerabilidades. \cite{api_tpl_zhang}

Adicionalmente, foi observado que as ferramentas CogniCrypt e CryptoGuard, embora extremamente importantes em termos de sua capacidade de detectar possíveis vulnerabilidades, não são capazes de identificar de onde surgem os alertas, sejam eles originários de bibliotecas nativas ou externas. \cite{perception_developers} Tal limitação poderia potencialmente acarretar em falsos positivos ou negligenciar alertas de importância vital advindos de bibliotecas de fundamento.

Para superar esse desafio, lançamos mão do estudo intitulado "Automated Third-Party Library Detection for Android Applications: Are We There Yet?". \cite{perception_developers} A partir dessa fonte, propomos uma solução inovadora ao integrar o resultado do LibScout ao contexto do CryptoGuard e CogniCrypt. Esta abordagem possibilitou não apenas a detecção de potenciais vulnerabilidades, mas também a identificação precisa de correspondências associadas a bibliotecas externas. Desse modo, concebeu-se uma flag adicional, denominada "external\_library", destinada a sinalizar a presença de uma biblioteca externa quando uma correspondência era identificada.

Também foi considerado o mapeamento geral das bibliotecas encontradas nos resultados da ferramenta LibScout \cite{LibScout} o que nos possibilitou identificar não só as bibliotecas que definitivamente eram externas como também fazer o casamento das classes apresentadas pelos analisadores estáticos surgindo assim outra flag denominada "possible\_external". Esta destinada a sinalizar se a biblioteca continha classes que poderiam ser externas. 

No entanto, é essencial mencionar os desafios enfrentados ao usar o LibScout. Por vezes, a ferramenta apresentou limitações ao definir clusters com diferentes graus de granularidade. Como resultado, os resultados podem não incluir bibliotecas conhecidas como não-nativas. Além disso, o conjunto de dados mais recente que está disponível para uso data de julho de 2019, o que pode alterar a extensão das correspondências identificadas.

A inclusão deste recurso não apenas aumentou a precisão da detecção de falhas, mas também abriu novas perspectivas. Agora somos capazes de fornecer diretamente recomendações aos desenvolvedores das bibliotecas em questão, o que permite uma intervenção mais direta e eficaz na resolução de possíveis vulnerabilidades. Antes, os desenvolvedores precisavam se encarregar da tarefa.

Este trabalho representa um avanço significativo na promoção da segurança de aplicações baseadas em Java, com o objetivo de proteger sistemas vitais e dados sensíveis de ameaças cibernéticas. Para contribuir para um ecossistema digital mais resiliente e protegido, as práticas de segurança na implementação de APIs criptográficas serão fortalecidas por meio dessa abordagem qualitativa e da integração de ferramentas de detecção.

\section{Objetivos}

O objetivo inicial deste estudo era fornecer aos desenvolvedores uma forma de identificar vulnerabilidades em APIs criptográficas. No entanto, ao longo do estudo, percebeu-se que a detecção de vulnerabilidades em APIs criptográficas, valendo-se das ferramentas CogniCrypt e CryptoGuard, não era suficiente para identificar a origem dos alertas.

Dessa forma, o objetivo deste estudo foi ampliado para incluir a identificação da origem dos alertas. Esta expansão se revelou crucial, uma vez que a capacidade de precisamente determinar a origem de um alerta é de extrema importância para os desenvolvedores. Isso possibilita ações direcionadas e específicas para corrigir possíveis vulnerabilidades, economizando tempo e recursos valiosos no processo de desenvolvimento e garantindo a segurança efetiva das aplicações.

Para isso, foi necessário integrar o resultado do LibScout ao contexto do CryptoGuard e CogniCrypt. Esta abordagem possibilitou não apenas a detecção de potenciais vulnerabilidades, mas também a identificação precisa de correspondências associadas a bibliotecas externas, fornecendo uma visão clara da origem dos alertas e permitindo a implementação de soluções de segurança de forma eficiente e focalizada.

\section{Justificativa}

A crescente complexidade das aplicações Java, aliada à importância crítica da segurança da informação, torna imperativo o desenvolvimento de técnicas e ferramentas que auxiliem os desenvolvedores na identificação e correção de potenciais vulnerabilidades em APIs criptográficas. Diversos estudos demonstraram que o uso inadequado dessas APIs é uma das principais fontes de vulnerabilidades em software. 

Diante desse cenário, a presente pesquisa se propõe a aprimorar a detecção de vulnerabilidades em APIs criptográficas, proporcionando aos desenvolvedores uma solução mais abrangente e eficaz para garantir a segurança das aplicações Java. A integração dos resultados do LibScout às ferramentas CryptoGuard e CogniCrypt representa um avanço significativo, pois não apenas identifica potenciais vulnerabilidades, mas também localiza a origem desses alertas, permitindo uma intervenção mais precisa e efetiva por parte dos desenvolvedores. 

Portanto, este estudo se justifica pela necessidade premente de fortalecer a segurança das aplicações Java e pela contribuição inovadora que a abordagem proposta representa para esse fim.
