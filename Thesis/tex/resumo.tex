Este estudo apresenta uma abordagem inovadora para aprimorar a detecção de vulnerabilidades em APIs
criptográficas Java, visando fortalecer a segurança de aplicações baseadas nessa tecnologia. 
Para isso, integramos as ferramentas CogniCrypt e CryptoGuard com o LibScout, permitindo a 
identificação precisa da origem de warnings relacionados a bibliotecas externas. 
Essa abordagem qualitativa representa um avanço significativo na promoção da segurança em aplicações Java,
contribuindo para um ecossistema digital mais resiliente e protegido contra potenciais 
ameaças cibernéticas. Ao incorporar a identificação da origem dos warnings, também possibilitamos 
sugestões diretas aos desenvolvedores das bibliotecas, otimizando o processo de correção de 
vulnerabilidades. No entanto, enfrentamos desafios ao analisar código obfuscado e ao utilizar 
clusters e datasets no LibScout, evidenciando a necessidade de aprimoramentos nessa ferramenta. 
A integração proposta neste trabalho representa um passo significativo em direção à segurança 
abrangente de dados sensíveis e sistemas críticos em aplicações Java.