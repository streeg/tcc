Este estudo apresenta uma abordagem para aprimorar a detecção de vulnerabilidades em APIs
criptográficas Java, visando fortalecer a segurança de aplicações baseadas nessa tecnologia. 
Para isso, integramos as ferramentas CogniCrypt e CryptoGuard com a ferramenta LibScout, permitindo a 
identificação da origem de warnings relacionados a bibliotecas externas. 
Essa abordagem qualitativa representa um avanço na promoção da segurança em aplicações Java,
contribuindo para um ecossistema digital mais resiliente e protegido contra potenciais 
ameaças cibernéticas. Ao incorporar a identificação da origem dos warnings, também possibilitamos 
sugestões diretas aos desenvolvedores das bibliotecas, otimizando o processo de correção de 
vulnerabilidades. Após a análise de mais de 200 aplicativos Android, foi possível observar que para
a ferramenta Cognicrypt tivemos 6798 warnings totais com 2436 pertecendo a bibliotecas externas e para o
Cryptoguard tivemos 2710 warnings totais com 1394 pertecendo a bibliotecas externas.