\section{Estudo Exprimental}

Para alcançar o objetivo de identificar quantos warnings das ferramentas CogniCrypt e CryptoGuard advém ou não de uma biblioteca extena, adotamos a abordagem GQM (Goal, Question, Metric), conforme descrito a seguir:

\subsection{Objetivo}

Identificar a quantidade de \textit{warnings} das ferramentas CogniCrypt e CryptoGuard que advém de bibliotecas externas. 

\subsubsection{Questões de Pesquisa}
Definimos as seguintes perguntas baseadas nos dados coletados:

\begin{itemize}
\item \textbf{RQ1:} Qual é a quantidade de \textit{warnings} no \textit{dataset} de aplicativos Android analisados pelas ferramentas CogniCrypt e CryptoGuard?

\item \textbf{RQ2:} Qual a quantidade de \textit{warnings} específicos de bibliotecas externas?

\item \textbf{RQ3:} Dado as limitações de análise da ferramenta LibScout, qual a quantidade de \textit{warnings} que possivelmente são de bibliotecas externas?

\end{itemize}

\subsubsection{Métricas}
Para responder as questões RQ1, RQ2 e RQ3 as seguintes métricas foram estabelecidas:

\begin{itemize}
\item \textbf{M1: Número total de warnings detectados pela ferramenta CogniCrypt.} \
Nos 307 aplicativos analisados, após executar o CogniCrypt, a ferramenta identificou 11.036 \textit{warnings} de crypto api missuses.

\item \textbf{M2: Número total de warnings detectados pela ferramenta CriptoGuard.} \
Nos mesmos aplicativos analisados pelo CogniCrypt, a ferramenta CryptoGuard identificou 4.964 \textit{warnings}. Essas métricas são essenciais para quantificar a presença geral de vulnerabilidades no conjunto de dados de aplicativos Android.


\item \textbf{M3: Número total de bibliotecas externas na ferramenta CogniCrypt.} \

No CogniCrypt, de um total de 6.798 bibliotecas contabilizadas dos 307 aplicativos, a ferramenta LibScout identificou que 1.710 \textit{warnings} advém de bibliotecas externas.

\item \textbf{M4: Número total de bibliotecas externas na ferramenta CriptoGuard.} \

No CryptoGuard, de um total de 2.710 bibliotecas dos mesmos 307 aplicativos, o LibScout identificou que 1.149 \textit{warnings} advém de bibliotecas externas. 

Isso demonstra que uma parte considerável dos \textit{warnings} detectados pode ser atribuída a bibliotecas externas, o que sugere a importancia da integração dessas ferramentas na identificação da origem dos \textit{warnings} e na detecção de vulnerabilidades específicas de bibliotecas externas.

Com o resultado da integração do resultado do LibScout com os \textit{warnings} identificados pelo CogniCrypt e CriptoGuard criamos uma abordagem capaz de identificar os \textit{warnings} associados a bibliotecas externas daqueles associados a bibliotecas nativas.

\item \textbf{M5: Número total de bibliotecas possivelmente externas na ferramenta CogniCrypt} \

Após a integração dos resultados do LibScout com os \textit{warnings} identificados por ambas as ferramentas, observamos que apesar do LibScout gerar resultados com precisão de algumas bibliotecas externas, a ferramenta não é capaz de identificar todas as bibliotecas externas presentes nos aplicativos \cite{api_tpl_zhang}.

Para melhorar a precisão dos resultados, identificamos também as bibliotecas que estão presentes nos aplicativos, mas que não foram detectadas pelo LibScout como bibliotecas externas. Uma biblioteca reconhecida pelo LibScout e presente nos resultados das ferramentas de análise estática foram marcadas como possivelmente externas e foram incluídas na análise.

Das 6.798 bibliotecas identificadas pelo CogniCrypt, 726 foram marcadas como possivelmente externas.

\item \textbf{M6: Número total de bibliotecas possivelmente externas na ferramenta CriptoGuard} \

Já das 2.710 bibliotecas identificadas pelo CryptoGuard, 245 foram marcadas como possivelmente externas.


\end{itemize}

\section{Metodologia do Estudo}

Para conseguir essas métricas e responder às questões de pesquisa, a pesquisa foi dividida em várias fases, conforme descrito a seguir:

\begin{figure}[!ht]
  \centering
  \includegraphics[scale=0.4]{img/research_steps1.png}
  \caption{Fases da pesquisa}
  \label{img: research_steps}
\end{figure}

\FloatBarrier

\begin{figure}[!ht]
  \centering
  \includegraphics[scale=0.4]{img/research_steps2.png}
  \caption{Fases da pesquisa}
  \label{img: research_steps2}
\end{figure}

\FloatBarrier

\subsection{Seleção de dataset} 

Inicialmente, foi coletado um conjunto de aplicativos Android de código aberto para análise. O conjunto de dados foi obtido do repositório \href{https://f-droid.org/pt_BR/packages/}{F-Droid}, que contém aplicativos de código aberto disponíveis para download. O dataset foi separado em categorias: Connectivity, Finances, Internet, Security, SMS e System. O repositório foi escolhido devido à sua natureza de código aberto e à disponibilidade de aplicativos para download. 

\subsection{Análise de vulnerabilidades e alerta aos desenvolvedores} 

Em seguida, foram utilizadas as ferramentas CogniCrypt e CryptoGuard para analisar os aplicativos e identificar vulnerabilidades em APIs criptográficas. As ferramentas foram escolhidas devido à sua capacidade de detectar vulnerabilidades em APIs criptográficas e fornecer alertas aos desenvolvedores sobre possíveis problemas de segurança. Nessa etapa, identificamos que vários dos \textit{warnings} detectados pelas ferramentas estavam associados a código que os desenvolvedores não tinham escrito e sim a códigos de bibliotecas externas. Ambas as ferramentas foram executadas em um ambiente docker rodando scripts estruturados para garantir a reprodutibilidade dos resultados. 


\subsubsection{CogniCrypt}

\begin{figure}[!ht]
  \centering
  \includegraphics[scale=0.4]{img/cognicrypt_script.png}
  \caption{Script utilizado para rodar o CogniCrypt}
  \label{img: cognicrypt_script}
\end{figure}

\FloatBarrier

\begin{figure}[!ht]
  \centering
  \includegraphics[scale=0.4]{img/cognicrypt_output.png}
  \caption{Output da ferramenta Cognicrypt para o apk Ademar.bitac\_5}
  \label{img: cognicrypt_output}
\end{figure}

\FloatBarrier

\begin{figure}[!ht]
  \centering
  \includegraphics[scale=0.4]{img/cognicrypt_output2.png}
  \caption{Output da ferramenta Cognicrypt para o apk Ademar.bitac\_5}
  \label{img: cognicrypt_output2}
\end{figure}

\FloatBarrier


\subsubsection{CriptoGuard}

\begin{figure}[!ht]
  \centering
  \includegraphics[scale=0.4]{img/cryptoguard_script.png}
  \caption{Script utilizado para rodar o CryptoGuard}
  \label{img: cryptoguard_script}
\end{figure}

\FloatBarrier

\begin{figure}[!ht]
  \centering
  \includegraphics[scale=0.4]{img/cryptoguard_output.png}
  \caption{Output da ferramenta Cryptoguard para o apk Ademar.bitac\_5}
  \label{img: cryptoguard_output}
\end{figure}

\FloatBarrier

\subsection{Identificação e Seleção das Ferramentas} 

Para identificar as bibliotecas externas, fizemos uma análise da literatura e através do estudo realizado por \cite{api_tpl_zhang}, identificamos o LibScout como uma ferramenta eficaz para identificar bibliotecas em aplicativos Android. O LibScout foi escolhido devido à sua capacidade de identificar bibliotecas em aplicativos Android e fornecer informações detalhadas sobre as bibliotecas encontradas. O estudo cita outras ferramentas que poderiam ser utilizadas, como o LibRadar, LibID, libPecker. O LibRadar foi um dos aplicativos inicialmente considerados para podermos comparar com os resutlados do LibScout, no entanto, o número de bibliotecas externas encotradas em um subset dos quais já sabiamos os resultados não foi suficiente. O LibRadar compara os resultados da investigação com um dataset base, o qual não foi atualizado nos últimos 7 anos. Cada uma das ferramentas destacadas no paper de Zhang tem suas limitações, como a falta de atualizações de datasets, tempo exagerado para análise de apenas um aplicativo, baixa confiabilidade nos resultados. De todas as métricas descritas no artigo, o LibScout foi a que apresentava resultados mais confiáveis em tempo hábil para análise de um grande número de aplicativos. 

Para a execução do LibScout, foi necessário instalar o Android SDK, além de configurar o ambiente de desenvolvimento. O LibScout foi executado em um ambiente docker através de scripts estruturados para garantir a reprodutibilidade dos resultados.

\begin{figure}[!ht]
  \centering
  \includegraphics[scale=0.4]{img/libscout_script.png}
  \caption{Script utilizado para rodar o LibScout}
  \label{img: libscout_script}
\end{figure}

\FloatBarrier

\begin{figure}[!ht]
  \centering
  \includegraphics[scale=0.4]{img/libscout_output.png}
  \caption{Output da ferramenta LibScout para o apk Ademar.bitac\_5}
  \label{img: libscout_output}
\end{figure}

\FloatBarrier

\begin{figure}[!ht]
  \centering
  \includegraphics[scale=0.4]{img/libscout_output2.png}
  \caption{Output da ferramenta LibScout para o apk Ademar.bitac\_5}
  \label{img: libscout_output2}
\end{figure}

\FloatBarrier

As bibliotecas identificadas pelo LibScout que não foram marcadas como externas mas que apareceram nos resultados do CogniCrypt e do CryptoGuard foram marcadas como possivelmente externas e incluídas na análise.

\subsection{Integração das Ferramentas} 

Após a seleção, integramos os resultados do LibScout aos contextos das ferramentas CryptoGuard e CogniCrypt. Essa integração permitiu uma análise mais detalhada e contextualizada das vulnerabilidades encontradas, especialmente em bibliotecas externas. A integração se deu por scripts que cruzam os resultados das ferramentas de análise estática com os resultados do LibScout.

\begin{figure}[!ht]
  \centering
  \includegraphics[scale=0.4]{img/integration_script_cogni.png}
  \caption{Script utilizado para integrar os resultados do LibScout com os resultados do CogniCrypt}
  \label{img: integration_script}
\end{figure}

\FloatBarrier

\begin{figure}[!ht]
  \centering
  \includegraphics[scale=0.4]{img/integration_script_crypto.png}
  \caption{Script utilizado para integrar os resultados do LibScout com os resultados do CryptoGuard}
  \label{img: integration_script}
\end{figure}

\FloatBarrier

A estratégia para adicionar a flag de possible external foi semelhante. Para cada biblioteca identificada pelo LibScout que não foi marcada como externa, mas que apareceu nos resultados do CogniCrypt e do CryptoGuard, adicionamos a flag de possivelmente externa.


\subsection{Análise do Experimento} Durante a análise, geramos gráficos e tabelas para ilustrar a distribuição de \textit{warnings} e a prevalência de vulnerabilidades. Os resultados foram gerados utilizando scripts para contar o número de \textit{warnings} e bibliotecas externas e possivelmente externas identificadas.

\begin{figure}[!ht]
  \centering
  \includegraphics[scale=0.4]{img/output_script_counter.png}
  \caption{Script utilizado para gerar os arquivos de saída para análise}
  \label{img: output_script}
\end{figure}

\FloatBarrier


\section{RQ1. Quantidade de warnings encontrados pelas ferramentas CogniCrypt e CryptoGuard}

Foram analisados 307 aplicativos de 6 diferentes categorias do repositório de aplicativos de código aberto F-Droid. A execução do CogniCrypt reportou 195 \textit{warnings} de uso indevido de criptografia, enquanto o CryptoGuard reportou 298. A tabela abaixo mostra a quantidade de aplicativos analisados por categoria e a quantidade de \textit{warnings} reportados por cada ferramenta.

\begin{table}[!htbp]
  \centering
  \begin{tabular}{|c|c|c|c|}
  
    \textbf{Categoria}   & \textbf{Número de Aplicativos}   &  \textbf{CogniCrypt}     &  \textbf{CryptoGuard} \\ 
     Connectivity           & \num{58}                         &  \num{20}                    & \num{3}                     \\
Finances                & \num{90}                         &  \num{25}                    & \num{2}                     \\
Internet                 & \num{39}                         &  \num{7}                      &     \num{0}                  \\
Security                 & \num{47}                         &  \num{16}                    &     \num{1}                  \\
Sms-Phone            & \num{18}                         &  \num{10}                     &     \num{2}                 \\
System                  & \num{55}                        &   \num{34}                    &     \num{0}                  \\
\textbf{Total}        & \num{307}                      &   \num{112}                  &     \num{8}                   \\
\end{tabular}
    
  \caption{Aplicativos por categoria sem warning das ferramentas CogniCrypt e CryptoGuard}
\label{AplicativosSemWarning}
\end{table}

Como visto, o número de aplicativos sem \textit{warnings} no CogniCrypt é bem maior do que no CryptoGuard. A categoria Sistema é a que apresenta a maior diferença entre eles, com \num{34} aplicativos. As categorias Conectividade e Finanças também têm um número elevado de aplicativos sem \textit{warnings}, \num{20} e \num{25}, respectivamente.

Apesar disso, o número de \textit{warnings} encontrados pelo CogniCrypt é maior do que no CryptoGuard, conforme descrito na tabela abaixo.

\begin{table}[!htbp]
  \centering
  \begin{tabular}{|c|c|c|c|}
  
\textbf{Categoria}   & \textbf{CogniCrypt (a)}   &  \textbf{CryptoGuard (b)}     &  \textbf{Diferença (a - b)} \\ 
Connectivity           & \num{1768} (\num{16.02}\%)  &  \num{1124} (\num{22.64}\%)  & \num{644} (\num{10.61}\%) \\
Finances                &     \num{3087} (\num{27.97}\%)     &     \num{1687} (\num{33.98}\%)     &     \num{1400} (\num{23.06}\%)\\
Internet                 &     \num{3407} (\num{30.87}\%)     &     \num{916} (\num{18.45}\%)     &     \num{2491} (\num{41.02}\%)\\
Security                 &     \num{1780} (\num{16.13}\%)     &     \num{553} (\num{11.14}\%)     &     \num{1227} (\num{20.21}\%)\\
Sms-Phone            &     \num{428} (\num{3.88}\%)     &     \num{171} (\num{3.44}\%)     &     \num{257} (\num{4.23}\%)\\
System                  &     \num{566} (\num{5.13}\%)     &     \num{513} (\num{10.33}\%)     &     \num{53} (\num{0.87}\%)\\
\textbf{Total}                     &     \num{11036} (\num{100.00}\%)     &     \num{4964} (100.00\%)     &     \num{6072} (\num{100.00}\%)\\
\end{tabular}
    
  \caption{Warnings encontrados nas ferramentas CogniCrypt e CryptoGuard}
\label{AplicativosComWarning}
\end{table}

Como podemos observar na tabela acima, o CogniCrypt conseguiu encontrar \num{4964} \textit{warnings}, enquanto o CryptoGuard encontrou \num{11036} \textit{warnings}. A diferença numérica é de \num{6072} (ou \num{122.32}\%) \textit{warnings} entre as duas ferramentas. As categorias de Finanças e Internet concentraram \num{58.8}\% (\num{6494}) dos \textit{warnings} do CogniCrypt, enquanto as categorias de Finanças e Conectividade concentraram \num{56.6}\% (\num{2811}) dos \textit{warnings} do CryptoGuard. A maior diferença entre os \textit{warnings} encontrados foi notada nas categorias Internet (\num{2491}) e Finanças (\num{1400}), representando \num{64.1}\% de diferença.

\section{RQ2 e RQ3. Quantidade de warnings de bibliotecas externas e possivelmente externas}

Em cada categoria, serão exibidos os resultados relativos ao número de aplicativos com alertas de vulnerabilidade, diferenciando aqueles que são potencialmente externos dos que são definitivamente externos, para cada ferramenta utilizada. As tabelas a seguir apresentam os resultados dessas integrações.

\textbf{CogniCrypt}

\begin{table}[!htbp]
  \centering
  \small
  \begin{tabular}{|c|c|c|c|c|}
  
\textbf{CC/Connectivity}   & \textbf{Warnings (a)}   &  \textbf{Possible ext (b)}     &  \textbf{Definite ext (c)} &  \textbf{Native-ext (a-b-c)} \\ 
Média                      & \num{21.9}              &  \num{2.3}                                         & \num{2.4}                                        & \num{17.1}                                                    \\
Desvio Padrão              & \num{43.8}              &  \num{10.3}                                         & \num{6.1}                                        & \num{42.8}                                 \\                    
Variância                  & \num{1919.8}            &  \num{107.8}                                         & \num{38.2}                                       & \num{1831.9}         \\                                           
\end{tabular}
    
  \caption{Resultados da integração do CogniCrypt com o LibScout na categoria Connectivity}
\label{table: AplicativosComWarningCCC}
\end{table}


\begin{table}[!htbp]
  \centering
  \small
  \begin{tabular}{|c|c|c|c|c|}
  
\textbf{CC/Finances}   & \textbf{Warnings (a)}   &  \textbf{Possible ext (b)}     &  \textbf{Definite ext (c)} &  \textbf{Native-ext (a-b-c)} \\ 
Média                      & \num{46.4}          &  \num{2.8}                                                  & \num{7.5}                                         & \num{35.9}                                                    \\
Desvio Padrão              & \num{132.06}           &  \num{10.8}                                                 & \num{41.8}                                        & \num{126.07}          \\                                          
Variância                  & \num{17439.9}       &  \num{118.2}                                                & \num{1747.7}                                      & \num{15895.3}    \\                                                
\end{tabular}
    
  \caption{Resultados da integração do CogniCrypt com o LibScout na categoria Finances}
\label{table: AplicativosComWarningCCF}
\end{table}


\begin{table}[!htbp]
  \centering
  \small
  \begin{tabular}{|c|c|c|c|c|}
  
\textbf{CC/SMS}   & \textbf{Warnings (a)}   &  \textbf{Possible ext (b)}     &  \textbf{Definite ext (c)} &  \textbf{Native-ext (a-b-c)} \\ 
Média                      & \num{11.7}              &  \num{2.36}                                         & \num{1}                                        & \num{8.3}                                                    \\
Desvio Padrão              & \num{31.67}              &  \num{7.78}                                         & \num{3.69}                                        & \num{24.1}   \\                                                 
Variância                  & \num{1003.03}            &  \num{60.54}                                         & \num{12.94}                                       & \num{585.4}             \\                                       
\end{tabular}
    
  \caption{Resultados da integração do CogniCrypt com o LibScout na categoria SMS}
\label{table: AplicativosComWarningCCSMS}
\end{table}


\begin{table}[!htbp]
  \centering
  \small
  \begin{tabular}{|c|c|c|c|c|}
  
\textbf{CC/System}   & \textbf{Warnings (a)}   &  \textbf{Possible ext (b)}     &  \textbf{Definite ext (c)} &  \textbf{Native-ext (a-b-c)} \\ 
Média                      & \num{10.09}              &  \num{0.93}                                         & \num{0.36}                                        & \num{8.79}                                                    \\
Desvio Padrão              & \num{33.01}              &  \num{3.74}                                         & \num{1.28}                                        & \num{31.82}  \\                                                  
Variância                  & \num{1090.2}            &  \num{14.05}                                         & \num{1.66}                                       & \num{1012.5}  \\                                                  
\end{tabular}
    
  \caption{Resultados da integração do CogniCrypt com o LibScout na categoria System}
\label{table: AplicativosComWarningCCS}
\end{table}

Os resultados para a tabela de conectividade (\ref{table: AplicativosComWarningCCC}) mostram que, em média, por aplicativo, o CogniCrypt encontrou \num{21.9} \textit{warnings} de vulnerabilidade. Desses, \num{2.3} são possivelmente externos e \num{2.4} são definitivamente externos. A quantidade de \textit{warnings} nativos é de \num{17.1}.
Para finanças (\ref{table: AplicativosComWarningCCF}), a média de \textit{warnings} por aplicativo é de \num{46.4}. Desses, \num{2.8} são possivelmente externos e \num{7.5} são definitivamente externos. A quantidade de \textit{warnings} nativos é de \num{35.9}. 
Para SMS (\ref{table: AplicativosComWarningCCSMS}), a média de \textit{warnings} por aplicativo é de \num{11.7}. Desses, \num{2.36} são possivelmente externos e \num{1} é definitivamente externo. A quantidade de \textit{warnings} nativos é de \num{8.3}.
E para sistema (\ref{table: AplicativosComWarningCCS}), a média de \textit{warnings} por aplicativo é de \num{10.09}. Desses, \num{0.93} são possivelmente externos e \num{0.36} são definitivamente externos. A quantidade de \textit{warnings} nativos é de \num{8.79}.
Em todos os exemplos, o desvio padrão e a variância são altos, indicando que os valores estão bem dispersos. Isso é explicado tanto pelas limitações do LibScout quanto pelas limitações do CogniCrypt. O LibScout pode não ter mapeado a biblioteca com \textit{warning} como externa, e o CogniCrypt pode ter encontrado \textit{warnings} em bibliotecas que não foram mapeadas pelo LibScout ou ainda não ter encontrado vulnerabilidade no aplicativo selecionado.

\textbf{CryptoGuard}

\begin{table}[!htbp]
  \centering
  \small
  \begin{tabular}{|c|c|c|c|c|}
    \hline
    \textbf{CG/Connectivity} & \textbf{Total Libraries} & \textbf{Possible Ext.} & \textbf{Definite Ext.} & \textbf{Native Libraries} \\
    \hline
    Média & \num{15.1} & \num{0.73} & \num{5.21} & \num{9.22} \\
    Desvio Padrão & \num{25.62} & \num{2.17} & \num{9.91} & \num{18.7} \\
    Variância & \num{656.6} & \num{4.71} & \num{98.3} & \num{352.4} \\
    \hline
  \end{tabular}
  \caption{Resultados da integração do CryptoGuard com o LibScout na categoria Connectivity}
  \label{table: AplicativosComWarningCGC}
\end{table}


\begin{table}[!htbp]
  \centering
  \small
  \begin{tabular}{|c|c|c|c|c|}
    \hline
    \textbf{CG/Finances} & \textbf{Total Libraries} & \textbf{Possible Ext.} & \textbf{Definite Ext.} & \textbf{Native Libraries} \\
    \hline
    Média & \num{10.45} & \num{1.33} & \num{4.68} & \num{4.43} \\
    Desvio Padrão & \num{20.71} & \num{4.72} & \num{10.03} & \num{10.05} \\
    Variância & \num{429.13} & \num{22.2} & \num{100.72} & \num{101.1} \\
    \hline
  \end{tabular}
  \caption{Resultados da integração do CryptoGuard com o LibScout na categoria Finances}
  \label{table: AplicativosComWarningCGF}
\end{table}


\begin{table}[!htbp]
  \centering
  \small
  \begin{tabular}{|c|c|c|c|c|}
    \hline
    \textbf{CG/SMS} & \textbf{Total Libraries} & \textbf{Possible Ext.} & \textbf{Definite Ext.} & \textbf{Native Libraries} \\
    \hline
    Média & \num{9} & \num{0.78} & \num{2.73} & \num{5.47} \\
    Desvio Padrão & \num{18.72} & \num{2.09} & \num{5.66} & \num{16.22} \\
    Variância & \num{350.6} & \num{4.37} & \num{32.08} & \num{263.19} \\
    \hline
  \end{tabular}
  \caption{Resultados da integração do CryptoGuard com o LibScout na categoria SMS}
  \label{table: AplicativosComWarningCGSMS}
\end{table}


\begin{table}[!htbp]
  \centering
  \small
  \begin{tabular}{|c|c|c|c|c|}
    \hline
    \textbf{CG/System} & \textbf{Total Libraries} & \textbf{Possible Ext.} & \textbf{Definite Ext.} & \textbf{Native Libraries} \\
    \hline
    Média & \num{7.53} & \num{0.65} & \num{2.67} & \num{4.2} \\
    Desvio Padrão & \num{16.49} & \num{2.69} & \num{7.07} & \num{12} \\
    Variância & \num{272.21} & \num{7.27} & \num{50.07} & \num{146.4} \\
    \hline
  \end{tabular}
  \caption{Resultados da integração do CryptoGuard com o LibScout na categoria System}
  \label{table: AplicativosComWarningCGS}
\end{table}

Analisando a tabela de conectividade (\ref{table: AplicativosComWarningCGC}), observa-se que o CryptoGuard, em média por aplicativo, detectou \num{15.1} alertas de vulnerabilidade, dos quais \num{0.73} são potencialmente externos e \num{5.21} definitivamente externos, com \num{9.22} alertas nativos. 
Na categoria de finanças (\ref{table: AplicativosComWarningCGF}), a média foi de \num{10.45} alertas por aplicativo, com \num{1.33} potencialmente externos e \num{4.68} definitivamente externos, além de \num{4.43} nativos.
Para SMS (\ref{table: AplicativosComWarningCGSMS}), a média foi de \num{9} alertas, com \num{0.78} potencialmente externos e \num{2.73} definitivamente externos, e \num{5.47} nativos.
E para sistema (\ref{table: AplicativosComWarningCGS}), a média foi de \num{7.53} alertas, com \num{0.65} potencialmente externos e \num{2.67} definitivamente externos, além de \num{4.2} nativos. 
Os altos desvios padrão e variância indicam uma dispersão significativa, influenciada pelas limitações do LibScout e do CryptoGuard. O LibScout pode não ter mapeado a biblioteca com \textit{warning} como externa, e o CryptoGuard pode ter encontrado \textit{warnings} em bibliotecas que não foram mapeadas pelo LibScout ou ainda não ter encontrado vulnerabilidade no aplicativo selecionado.


\begin{table}[!htbp]
  \centering
  \small
  \begin{tabular}{|c|c|c|}
    \hline
    \textbf{-} & \textbf{CogniCrypt} & \textbf{Cryptoguard} \\
    \hline
    Total Aplicativos & \num{246} & \num{253} \\
    Total Bibliotecas & \num{6798} & \num{2710}  \\
    Total Bibliotecas Externas & \num{1710} & \num{1149} \\
    Total Bibliotecas Externas Potenciais & \num{726} & \num{245} \\
    Total Bibliotecas Nativas & \num{4362} & \num{1316} \\
    \hline
  \end{tabular}
  \caption{Resultados da integração do CryptoGuard com o LibScout na categoria System}
  \label{table: AplicativosComWarningSummary}
\end{table}

A tabela \ref{table: AplicativosComWarningSummary} mostra um resumo dos resultados obtidos nas categorias analisadas. O CogniCrypt analisou 246 aplicativos, com 6798 bibliotecas, das quais 1710 são externas, 726 potencialmente externas e 4362 nativas. O CryptoGuard analisou 253 aplicativos, com 2710 bibliotecas, das quais 1149 são externas, 245 potencialmente externas e 1316 nativas.

\begin{figure}[!ht]
  \centering
  \includegraphics[scale=0.5]{img/plot_cc_x_cg_summary.png}
  \caption{Comparação total entre as ferramentas CogniCrypt e CryptoGuard}
  \label{img: CCvsCG_Summary}
\end{figure}

\FloatBarrier

\begin{figure}[!ht]
  \centering
  \includegraphics[scale=0.5]{img/plot_cc_x_cg_proportion_summary.png}
  \caption{Comparação proporcional total entre as ferramentas CogniCrypt e CryptoGuard}
  \label{img: CCvsCG_Summary2}
\end{figure}

\FloatBarrier

Os resultados para a ferramenta CogniCrypt se destacaram em relação aos resultados para o CryptoGuard. 

As figuras subsequentes ilustram uma análise comparativa categorizada entre as ferramentas CogniCrypt e CryptoGuard

\begin{figure}[!ht]
  \centering
  \includegraphics[scale=0.5]{img/plot_cc_x_cg_connectivity.png}
  \caption{Comparação entre as ferramentas CogniCrypt e CryptoGuard na categoria Connectivity}
  \label{img: CCvsCG_Connectivity}
\end{figure}

\FloatBarrier

Na categoria de conectividade (\ref{img: CCvsCG_Connectivity}), a média de alertas emitidos pelas ferramentas indica que o CogniCrypt gera um número maior de alertas por aplicativo, incluindo alertas nativos e potencialmente externos. Contudo, o CryptoGuard excede no número de alertas definitivamente classificados como externos.
O CogniCrypt apresenta um desvio padrão e uma variância superiores, com exceção da quantidade de alertas de bibliotecas categorizadas como definitivamente externas.
A eficiência na integração com o LibScout, para a identificação de bibliotecas definitivamente externas, é mais pronunciada no CryptoGuard.

\begin{figure}[!ht]
    \centering
    \includegraphics[scale=0.5]{img/plot_cc_x_cg_finances.png}
    \caption{Comparação entre as ferramentas CogniCrypt e CryptoGuard na categoria Finances}
    \label{img: CCvsCG_Finances}
\end{figure}

\FloatBarrier

Na categoria financeira  (\ref{img: CCvsCG_Finances}), o CogniCrypt demonstrou superioridade em relação ao CryptoGuard.
Isso indica que o CryptoGuard tem uma dispersão maior nos valores relativos aos alertas por aplicativo e aos alertas de bibliotecas nativas, em contraste com a maior dispersão do CogniCrypt nos valores de alertas possivelmente de bibliotecas externas e definitivamente externas.

\begin{figure}[!ht]
    \centering
    \includegraphics[scale=0.5]{img/plot_cc_x_cg_sms.png}
    \caption{Comparação entre as ferramentas CogniCrypt e CryptoGuard na categoria SMS}
    \label{img: CCvsCG_SMS}
\end{figure}

\FloatBarrier

\begin{figure}[!ht]
    \centering
    \includegraphics[scale=0.5]{img/plot_cc_x_cg_system.png}
    \caption{Comparação entre as ferramentas CogniCrypt e CryptoGuard na categoria System}
    \label{img: CCvsCG_System}
\end{figure}


\FloatBarrier

Nas categorias de SMS (\ref{img: CCvsCG_SMS}) e Sistemas (\ref{img: CCvsCG_System}), observa-se um padrão análogo ao da categoria de conectividade.
O CogniCrypt gera uma quantidade superior de alertas por aplicativo, incluindo uma maior frequência de alertas nativos e potencialmente externos.
Em contrapartida, o CryptoGuard excede no número de alertas categorizados como definitivamente externos.
Esta tendência também se reflete no desvio padrão e na variância, onde o CogniCrypt mostra maior dispersão de dados, à exceção dos alertas definitivamente externos.
